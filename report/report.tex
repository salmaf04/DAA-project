\documentclass[12pt]{article}
\usepackage[utf8]{inputenc}
\usepackage[spanish]{babel}
\usepackage{amsmath}
\usepackage{amssymb}
\usepackage{geometry}

\geometry{a4paper, margin=1in}

\title{Proyecto de Diseño y Análisis de Algoritmos: Problema de Logística Discreta (DLP)}
\author{Salma Fonseca Curbelo C-412, José Ernesto Morales Lazo C-412}
\begin{document}

\maketitle

\section{Formalización del Modelo de Logística Discreta}

\subsection{Estructuras de Datos de Entrada}
\begin{itemize}
    \item \textbf{Mulas ($M$):} Un arreglo \texttt{mulas[]} de tamaño $n$, donde cada posición $i$ contiene la capacidad $c_i$.
    \begin{itemize}
        \item $c_i \in \mathbb{R}^+$: Capacidad de carga de la mula $m_i$.
    \end{itemize}
    \item \textbf{Artículos ($A$):} Un arreglo \texttt{articulos[]} de tamaño $k$, donde cada posición $j$ contiene una tupla $\langle w_j, v_j \rangle$.
    \begin{itemize}
        \item $w_j \in \mathbb{R}^+$: Peso del artículo $a_j$.
        \item $v_j \in \mathbb{R}^+$: Valor del artículo $a_j$.
    \end{itemize}
\end{itemize}

\subsection{Variables de Decisión}
Definimos la matriz binaria de asignación:
\[
x_{ij} = 
\begin{cases} 
1 & \text{si el artículo } j \text{ se asigna a la mula } i \\ 
0 & \text{en otro caso} 
\end{cases}
\]
$\forall i \in \{1, \dots, n\}, \forall j \in \{1, \dots, k\}$.

\subsection{Modelo Matemático (Linealizado)}
Para que el modelo sea procesable por un computador, linealizamos la función objetivo utilizando dos variables auxiliares: $V_{max}$ (valor de la mula más cargada) y $V_{min}$ (valor de la mula menos cargada).

\textbf{Función Objetivo:}
\[ \text{Minimizar } Z = V_{max} - V_{min} \]

\textbf{Sujeto a (Restricciones):}
\begin{enumerate}
    \item \textbf{Cálculo de Valores por Mula:}
    El valor total $V_i$ transportado por la mula $i$ se define como:
    \[ V_i = \sum_{j=1}^{k} v_j x_{ij} \quad \forall i \in \{1, \dots, n\} \]
    Y debe cumplirse que:
    \[ V_{min} \le V_i \le V_{max} \quad \forall i \in \{1, \dots, n\} \]

    \item \textbf{Asignación Única (Partición):}
    Cada artículo $j$ debe asignarse exactamente a una mula.
    \[ \sum_{i=1}^{n} x_{ij} = 1 \quad \forall j \in \{1, \dots, k\} \]

    \item \textbf{Capacidad de Peso:}
    La carga de la mula $i$ no debe exceder su capacidad individual $c_i$.
    \[ \sum_{j=1}^{k} w_j x_{ij} \le c_i \quad \forall i \in \{1, \dots, n\} \]

    \item \textbf{Integridad de Variables:}
    \[ x_{ij} \in \{0, 1\}, \quad V_{max}, V_{min} \ge 0 \]
\end{enumerate}

\subsection{Propiedades de la Salida}
\begin{itemize}
    \item \textbf{Factibilidad:} Si no existe ninguna combinación de $x_{ij}$ que satisfaga simultáneamente la capacidad de peso de todas las mulas y la asignación de todos los artículos, el sistema debe informar que la instancia es Irresoluble.
    \item \textbf{Optimidad:} La solución devuelta debe garantizar que no existe otra configuración donde la brecha de riesgo $(V_{max} - V_{min})$ sea menor.
    \item \textbf{Formato de Salida:} Una lista de conjuntos $S = \{M_1, M_2, \dots, M_n\}$, donde cada $M_i$ contiene los índices $j$ de los artículos tales que $x_{ij} = 1$.
\end{itemize}

\section{Análisis de Complejidad Computacional}

\subsection{Demostración de que el Problema de la Partición es NP Completo}

\textbf{El Problema de la Partición:} Dado un multiconjunto $S$ de enteros positivos , ¿es posible dividir $S$ en dos subconjuntos $S_1$ y $S_2$ tales que la suma de los elementos en $S_1$ sea igual a la suma de los elementos en $S_2$?

\subsubsection{Demostración de que el problema de la partición es NP:}

Un problema está en la clase NP si una solución propuesta (un "certificado") puede ser verificada en tiempo polinomial.

\begin{enumerate}
    \item \textbf{Certificado:} Una lista de los elementos que pertenecen al subconjunto $S_1$.
    \item \textbf{Verificación:} Un algoritmo simplemente suma los elementos de $S_1$ y suma los elementos de $S \setminus S_1$ (que sería $S_2$). Luego compara si las sumas son iguales.
    \item \textbf{Tiempo:} La suma y la comparación son operaciones lineales. Por lo tanto, el problema de la partición es NP.
\end{enumerate}

\subsubsection{Demostración de que el problema de la partición es NP-Completo:}

Se usará el problema de la Suma de Subconjuntos (Subset-Sum), el cual se vió en clase que es NP-Completo.

\textbf{Subset-Sum:} Dado un conjunto de enteros $A$ y un entero objetivo $t$, ¿existe un subconjunto $A' \subseteq A$ tal que la suma de sus elementos sea $t$?

\textbf{Reducción Subset-Sum $\le_p$ Partición}

Se debe transformar cualquier instancia de Subset-Sum en una instancia del problema de la Partición en tiempo polinomial.

\begin{enumerate}
    \item \textbf{Construcción:}
    Sea una instancia de Subset-Sum con conjunto $A$ y objetivo $t$.
    Se calcula la suma total de $A$, denotada como $\Sigma = \Sigma_{i=1}^n a_i$.
    Se construye un nuevo conjunto $S$ para el problema partición agregando dos elementos adicionales especiales, $J$ y $K$, al conjunto $A$.
    \begin{itemize}
        \item $J = 2\Sigma - t$
        \item $K = \Sigma + t$
        \item $S = A \cup \{J, K\}$
    \end{itemize}

    \item \textbf{Equivalencia:}
    La suma total de los elementos en el nuevo conjunto es:
    \[ \text{Sum}(S) = \Sigma + (2\Sigma - t) + (\Sigma + t) = 4\Sigma \]
    
    Para que exista una partición válida en $S$, se debe dividir $S$ en dos subconjuntos, cada uno sumando exactamente la mitad del total, es decir, $2\Sigma$.
    
    Cómo se puede llegar a $2\Sigma$:
    \begin{itemize}
        \item Los elementos $J$ y $K$ no pueden estar en el mismo subconjunto porque $J + K = 3\Sigma$, lo cual excede el objetivo $2\Sigma$.
        \item Por lo tanto, $J$ debe estar en un subconjunto ($S_1$) y $K$ en el otro ($S_2$).
        \item Para que $S_1$ sume $2\Sigma$, debe contener a $J$ más algunos elementos de $A$ ($A_{sub}$) tal que:
        \[ J + \Sigma A_{sub} = 2\Sigma \]
        \[ (2\Sigma - t) + \Sigma A_{sub} = 2\Sigma \]
        \[ \Sigma A_{sub} = t \]
    \end{itemize}
\end{enumerate}

\textbf{Conclusión:} Existe una partición válida en $S$ si y solo si existe un subconjunto en $A$ que sume exactamente $t$. Como la transformación es polinomial, el problema de la Partición es NP-Completo.

\subsection{Demostración de que el Problema de la DLP es NP Completo}

\subsubsection{Demostración de que DLP es NP:}

\begin{enumerate}
    \item \textbf{Certificado:} Una asignación propuesta de cada artículo a una mula específica.
    \item \textbf{Verificación:} Un algoritmo verificador calcula la suma de pesos de cada una de las $m$ mulas y comprueba si excede el límite de peso $C$. Luego calcula la suma de valores $V_j$ para cada mula y compara para cada par de mulas $V_a$ y $V_b$ y verifica si $|V_a-V_b| \le K$.
    \item \textbf{Tiempo:} Las sumas son operaciones lineales $O(n)$ y como hay $m(m-1)/2$ pares de mulas y $m \le n$, la compración es $O(n^2)$, por lo que se puede verificar en tiempo polinomial.
\end{enumerate}

Como la verificación se realiza en tiempo polinomial, DLP es NP.

\subsubsection{Demostración que DLP es NP-difícil}

Se usará el problema de la Partición, el cual se demostró anteriormente que es NP-Completo.

\textbf{Reducción Partición $\le_p$ DLP}

Se tomará una instancia arbitraria de Partición con el conjunto $S$. Sea $W_{total} = \Sigma_{i=1}^n x_i$.
Se construye una instancia de DLP con los siguientes parámetros específicos:

\begin{enumerate}
    \item \textbf{Artículos:} Para cada número $x_i \in S$, se crea un artículo $i$.
    \item \textbf{Pesos y Valores:} Se define tanto el peso como el valor igual al número original: $w_i=x_i, v_i=x_i$.
    \item \textbf{Número de Mulas ($m$):} Se fija $m=2$.
    \item \textbf{Diferencia permitida ($K$):} Se fija $K=0$.
    \item \textbf{Capacidad ($C$):} Se fija $C = W_{total} / 2$. (Si $W_{total}$ es impar, no hay solución para Partición ni para este DLP, así que se asume par).
\end{enumerate}

\textbf{Demostración de Equivalencia:}

Si Partición tiene solución:
Existen $S_1$ y $S_2$ tales que $\Sigma S_1 = \Sigma S_2 = W_{total} / 2$.
Se asignan los artículos correspondientes a $S_1$ a la Mula 1 y los de $S_2$ a la Mula 2.
Verificación de Peso: La Mula 1 lleva peso $W_{total} / 2$, que es $= C$. La Mula 2 igual. (Cumple).
Verificación de Balance: El valor en la Mula 1 es $W_{total} / 2$. El valor en la Mula 2 es $W_{total} / 2$. La diferencia es $|W_{total} / 2 - W_{total} / 2| = 0$. Como $0 \le K$ (donde $K=0$), cumple la condición.
Por tanto, DLP devuelve SÍ.

Si DLP devuelve SÍ:
Significa que existe una distribución en 2 mulas tal que la diferencia de valores es $\le 0$. Como el valor absoluto no puede ser negativo, la diferencia debe ser exactamente 0.

\[ \Sigma_{i \in m1} v_i = \Sigma_{i \in m2} v_i \]

Dado que se definió $v_i = x_i$, esto implica:

\[ \Sigma_{i \in m1} x_i = \Sigma_{i \in m2} x_i \]

Esto constituye una partición válida del conjunto original en dos sumas iguales.

\textbf{Conclusión:}
Se ha demostrado que el problema Partición es un caso particular del problema DLP (específicamente cuando $m=2, w_i=v_i$ y $K=0$).
Dado que el caso particular es NP-Completo, el caso general es al menos igual de difícil.

Por lo tanto, el Problema de Logística Discreta (DLP) es NP-Completo.

\end{document}